\documentclass{pracamgr}

\usepackage{polski}
\usepackage{graphicx}
\usepackage{todonotes}
\usepackage{hyperref}

\usepackage[utf8]{inputenc}

% Dane magistranta:

\author{Marcel Kołodziejczyk}

\nralbumu{219533}

\title{Luki w bezpieczeństwie systemu operacyjnego Android}

\tytulang{Vulnerabilities in Android operating system}

%kierunek: Matematyka, Informatyka, ...
\kierunek{Informatyka}

% informatyka - nie okreslamy zakresu (opcja zakomentowana)
% matematyka - zakres moze pozostac nieokreslony,
% a jesli ma byc okreslony dla pracy mgr,
% to przyjmuje jedna z wartosci:
% {metod matematycznych w finansach}
% {metod matematycznych w ubezpieczeniach}
% {matematyki stosowanej}
% {nauczania matematyki}
% Dla pracy licencjackiej mamy natomiast
% mozliwosc wpisania takiej wartosci zakresu:
% {Jednoczesnych Studiow Ekonomiczno--Matematycznych}

% \zakres{Tu wpisac, jesli trzeba, jedna z opcji podanych wyzej}

% Praca wykonana pod kierunkiem:
% (podać tytuł/stopień imię i nazwisko opiekuna
% Instytut
% ew. Wydział ew. Uczelnia (jeżeli nie MIM UW))
\opiekun{dra Marcina Peczarskiego\\
  Instytut Informatyki\\
  }

% miesiąc i~rok:
\date{Czerwiec 2013}

%Podać dziedzinę wg klasyfikacji Socrates-Erasmus:
\dziedzina{
%11.0 Matematyka, Informatyka:\\
%11.1 Matematyka\\
%11.2 Statystyka\\
11.3 Informatyka\\
%11.4 Sztuczna inteligencja\\
%11.5 Nauki aktuarialne\\
%11.9 Inne nauki matematyczne i informatyczne
}

%Klasyfikacja tematyczna wedlug AMS (matematyka) lub ACM (informatyka)
\klasyfikacja{D. Software\\
  D.4. Operating Systems\\
  D.4.6. Security and Privacy Protection}

% Słowa kluczowe:
\keywords{android, arm, atak, bezpieczeństwo, przepełnienie bufora, metasploit, exploit, shellcode}

% Tu jest dobre miejsce na Twoje własne makra i~środowiska:
\newtheorem{defi}{Definicja}[section]

% koniec definicji

\begin{document}
\maketitle


%tu idzie streszczenie na strone poczatkowa
\begin{abstract}
krótkie streszczenie pracy
\end{abstract}


\tableofcontents
%\listoffigures
%\listoftables
\listoftodos

\chapter*{Wprowadzenie}

\chapter{Platforma sprzętowa i programowa}

Android jest systemem operacyjnym i zestawem aplikacji dedykowanym przede wszystkim dla urządzeń przenośnych z ekranami dotykowymi, takimi jak np. smartphone, tablet. 
Jądro systemu, zostało oparte na jądrze Linuksa. System ten został zaprojektowany i stworzony głównie z myślą o urządzeniach
wyposażonych w procesor w architekturze ARM, aczkolwiek podejmowane są prace nad dostosowaniem Androida do innych architektur, np. x86.

W rozdziale tym zostaną opisane podstawy architektury procesorów ARM. Następnie zostanie omówiona architektura oraz model bezpieczeństwa systemu Android.

\section{Architektura procesorów ARM}

ARM jest 32-bitową architekturą procesorów typu RISC. Główne jej cechy to:
\begin{itemize}
\item proste tryby adresowania
\item instrukcje stałej długości co ułatwia adresowanie
\item architektura typu \emph{load/store} - operacje wykonywane są na rejestrach a nie bezpośrednio na pamięci 
\item duża liczba 32-bitowych rejestrów
\item zredukowana liczba instrukcji
\end{itemize}


Z biegiem czasu ukazywały się kolejne wersje architektury ARM. Niniejsza praca bazuje na wersji 7 (ARMv7), które jest obecnie najbardziej 
powszechnie wykorzystywana w urządzeniach przenośnych.

\subsection{Thumb-2}

Instrukcje ARM są stałej, 32-bitowej długości. W celu zwiększenia gęstości kodu został wprowadzony drugi, uproszczony zestaw 16-bitowych instrukcji Thumb-2. 
Ponieważ w obydwu trybach adresy instrukcji muszą być odpowiednio wyrównane, ostatni bit adresu może być wykorzystany w celu zmiany trybu pracy procesora.
Instrukcja skoku do adresu, którego ostatni bit jest zapalony, wymusza zmianę trybu na Thumb-2 i dalsze wykonywanie instrukcji spod adresu odpowiednio wyrównanego.
Analogicznie instrukcja skoku do parzystego adresu powoduje przejście w 32-bitowy zestaw instrukcji ARM.

\subsection{Rejestry}

Z punktu widzenia programisty dostępnych jest szesnaście 32-bitowych rejestrów R0-R15. Trzy z nich mają dedykowane przeznaczenie:

\begin{itemize}
\item SP (Stack Pointer) - R13 - wskaźnik stosu
\item LR (Link Register) - R14 - zawiera adres następnej instrukcji przy instrukcjach skoku wywołania podprogramu
\item PC (Program Counter) - R15 - przechowuje adres następnej instrukcji
\end{itemize}

Dodatkowo występuje rejestr statusowy procesora CPSR (ang. Current Porcessor Status Register). Przechowuje on m. in. flagi Negative, Zero, Carry, oVerflow.
Większość instrukcji może być wykonywanych warunkowo, w zależności od stanu tych flag.

Szczegółowe informacje na ten temat można znaleźć w \cite{armman}.

\subsection{Standard wywołania procedur}

Do wykonywania procedur służą następujące instrukcje:
\begin{itemize}
\item \textbf{B} - Branch
\item \textbf{BL} - Branch with Link
\item \textbf{BX} - Branch and Exchange
\item \textbf{BLX} - Branch with Link and Exchange
\end{itemize}

Instrukcja \textit{Branch} umożliwia wykonanie skoku o maksymalnie 32 MB w przód lub w tył od bieżącej instrukcji. \textit{Branch with link} dodatkowo
zachowuje adres powrotu w rejestrze LR (R14). Pozostałe dwie instrukcje jako argument przyjmują rejestr - skok jest wykonywany do adresu, jaki znajduje się
w przekazanym rejestrze.

Architektura ARM określa następujące zasady wywoływania procedur:
\begin{itemize}
\item Do przekazywania argumentów i zwracania wyniku procedury używane są rejestry R0-R3. Kolejne parametry mogą być przekazywane na stosie.
\item rejestry R4-R11 mogą być wykorzystywana do przechowywania zmiennych lokalnych
\item Zawartość rejestrów R4-R12 powinna być zachowana w trakcie wykonania procedury. Zazwyczaj w prologu procedury rejestry te są odkładana na stos, 
aby przywrócić ich wartości w epilogu.
\item Stos rośnie w kierunku mniejszych adresów pamięci.
\end{itemize}

Kompletną dokumentację standardu wołania procedur można znaleźć w \cite{armcall}.

\section{Architektura systemu Android}





\section{Model bezpieczeństwa Androida}


\chapter{Przykłady ataków}
 % protokól wywołanie funkcji

\section{Klasyczny błąd przepełnienie bufora}

% przeciwdziałanie: Stack smashing protection

\section{Technika ,,heap spray''}

%przeciwdziałanie: DEP, NX bit, Executable space protection

\subsection{NX bit}

W celu ochrony przed tego typu takimi w wersji 6 architektury ARM wprowadzona została technologia NX bit (No Execute). Umożliwia ona systemowi operacyjnemu 
oznaczyć wybrane strony pamięci jako niewykonywalne. Gdy bit NX dla danej strony jest ustawiony, próba wykonania zawartości tej strony jako kodu kończy się 
wygenerowaniem wyjątku, zgłaszanego systemowi operacyjnemu, co powoduje przerwanie wykonywania programu. Bit NX powinien być ustawiony dla wszystkich stron procesu, 
z wyjątkiem programu i bibliotek oraz świadomie dozwolonych przez program wyjątków.

Technologia ,,NX bit'' jest wspierana przez Androida od wersji 2.3. 




 wyko dzięki któremu system operacyjny 

\section{Technika ,,return to library'' (Ret2Libc)}

% przeciwdziałanie: ASLR

\chapter{Tworzenie payloadów}


\chapter{Rozszerzenie Matesploita}

% ogolnie opisać matesploita, 
% materiały: 
%  praca dla samsunga
%  msf_user_guide.pdf

\section{CVE-2010-1119}

\section{CVE-2010-1807}

\chapter{Podsumowanie}

\begin{thebibliography}{99}
\addcontentsline{toc}{chapter}{Bibliografia}

% troche nie na temat
\bibitem[1]{decomp} Anthony Desnos, Geoffroy Gueguen \textit{Android: From Reversing to Decompilation}, Black Hat, Abu Dhabi, 2011 \

\bibitem[2]{privescal} S. Höbarth, R. Mayrhofer, \textit{A framework for on-device privilege escalation exploit execution on android}, IWSSI/SPMU 2011: 3rd International Workshop on Security and Privacy in Spontaneous Interaction and Mobile Phone Use, colocated with Pervasive 2011, czerwiec 2011. dostępne na \url{http://www.medien.ifi.lmu.de/iwssi2011/} \

\bibitem[3]{armguide} Gaurav Kumar, Aditya Gupta, \textit{A Short Guide on ARM Exploitation}, \url{http://www.exploit-db.com/wp-content/themes/exploit/docs/24493.pdf} \

\bibitem[4]{alphanum} Yves Younan, Pieter Philippaerts, \textit{Alphanumeric RISC ARM shellcode}, Phrack, 66, czerwiec 2009 \

\bibitem[5]{anatomy} Joshua Hulse, \textit{Buffer Overflows: Anatomy of an Exploit}, \url{http://packetstormsecurity.com/files/108549/Buffer-Overflows-Anatomy-Of-An-Exploit.html} \

\bibitem[6]{explarm} Emanuele Acri, \textit{Exploiting Arm Linux Systems}, \url{http://packetstormsecurity.com/files/98376/Exploiting-ARM-Linux-Systems.html} \

\bibitem[7]{fuzzphone} Collin Mulliner, Charlie Miller, \textit{Fuzzing the Phone in your Phone}, Black Hat USA, 2009 \

\bibitem[8]{howtoshell} Jonathan Salwan, \textit{How to Create a Shellcode on ARM Architecture}, \url{http://www.exploit-db.com/papers/15652/} \

\bibitem[9]{telephony} Dustin ,,I)ruid'' Trammel, \textit{Metasploit Framework Telephony}, Black Hat USA, 2009 \

\bibitem[10]{nonexec} Itzhak Avraham, \textit{Non-Executable Stack ARM Exploitation}, Black Hat DC, 2011 \

\bibitem[11]{smashing} jip@soldierx.com, \textit{Stack Smashing On A Modern Linux System}, \url{http://www.soldierx.com/tutorials/Stack-Smashing-Modern-Linux-System} \

\bibitem[12]{armman} ARM Ltd. \textit{Arm architecture reference manual.} \

\bibitem[13]{armcall} ARM Ltd. \textit{Procedure call standard for the arm architecture.} \

\bibitem[14]{metasploit} Metasploit framework, \url{http://www.metasploit.com} \

\bibitem[15]{android} Android project, \url{http://developer.android.com} \

\bibitem[16]{webkit} The WebKit Open Source Project, \url{http://www.webkit.org} \

\end{thebibliography}

\end{document}

