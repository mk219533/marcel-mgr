\documentclass{pracamgr}

\usepackage{polski}
\usepackage{graphicx}
\usepackage{todonotes}
\usepackage{hyperref}

\usepackage[utf8]{inputenc}

% Dane magistranta:

\author{Marcel Kołodziejczyk}

\nralbumu{219533}

\title{Luki w bezpieczeństwie systemu operacyjnego Android}

\tytulang{Vulnerabilities in Android operating system}

%kierunek: Matematyka, Informatyka, ...
\kierunek{Informatyka}

% informatyka - nie okreslamy zakresu (opcja zakomentowana)
% matematyka - zakres moze pozostac nieokreslony,
% a jesli ma byc okreslony dla pracy mgr,
% to przyjmuje jedna z wartosci:
% {metod matematycznych w finansach}
% {metod matematycznych w ubezpieczeniach}
% {matematyki stosowanej}
% {nauczania matematyki}
% Dla pracy licencjackiej mamy natomiast
% mozliwosc wpisania takiej wartosci zakresu:
% {Jednoczesnych Studiow Ekonomiczno--Matematycznych}

% \zakres{Tu wpisac, jesli trzeba, jedna z opcji podanych wyzej}

% Praca wykonana pod kierunkiem:
% (podać tytuł/stopień imię i nazwisko opiekuna
% Instytut
% ew. Wydział ew. Uczelnia (jeżeli nie MIM UW))
\opiekun{dra Marcina Peczarskiego\\
  Instytut Informatyki\\
  }

% miesiąc i~rok:
\date{Czerwiec 2013}

%Podać dziedzinę wg klasyfikacji Socrates-Erasmus:
\dziedzina{
%11.0 Matematyka, Informatyka:\\
%11.1 Matematyka\\
%11.2 Statystyka\\
11.3 Informatyka\\
%11.4 Sztuczna inteligencja\\
%11.5 Nauki aktuarialne\\
%11.9 Inne nauki matematyczne i informatyczne
}

%Klasyfikacja tematyczna wedlug AMS (matematyka) lub ACM (informatyka)
\klasyfikacja{D. Software\\
  D.4. Operating Systems\\
  D.4.6. Security and Privacy Protection}

% Słowa kluczowe:
\keywords{android, arm, atak, bezpieczeństwo, przepełnienie bufora, metasploit, exploit, shellcode}

% Tu jest dobre miejsce na Twoje własne makra i~środowiska:
\newtheorem{defi}{Definicja}[section]

% koniec definicji

\begin{document}
\maketitle


%tu idzie streszczenie na strone poczatkowa
\begin{abstract}
krótnie streszczenie pracy
\end{abstract}


\tableofcontents
%\listoffigures
%\listoftables
\listoftodos

\chapter*{Wprowadzenie}

\todo[noline]{Sprawdzić klasyfikację ACM}
Jakiś przydlugi niewiele mówiący opis.

\chapter{Omówienie problemu}

\chapter{Podsumowanie}

\begin{thebibliography}{99}
\addcontentsline{toc}{chapter}{Bibliografia}

% troche nie na temat
\bibitem[1]{decomp} Anthony Desnos, Geoffroy Gueguen \textit{Android: From Reversing to Decompilation}, Black Hat, Abu Dhabi, 2011 \

\bibitem[2]{privescal} S. Höbarth, R. Mayrhofer, \textit{A framework for on-device privilege escalation exploit execution on android}, IWSSI/SPMU 2011: 3rd International Workshop on Security and Privacy in Spontaneous Interaction and Mobile Phone Use, colocated with Pervasive 2011, czerwiec 2011. dostępne na \url{http://www.medien.ifi.lmu.de/iwssi2011/} \

\bibitem[3]{armguide} Gaurav Kumar, Aditya Gupta, \textit{A Short Guide on ARM Exploitation}, \url{http://www.exploit-db.com/wp-content/themes/exploit/docs/24493.pdf} \

\bibitem[4]{alphanum} Yves Younan, Pieter Philippaerts, \textit{Alphanumeric RISC ARM shellcode}, Phrack, 66, czerwiec 2009 \

\bibitem[5]{anatomy} Joshua Hulse, \textit{Buffer Overflows: Anatomy of an Exploit}, \url{http://packetstormsecurity.com/files/108549/Buffer-Overflows-Anatomy-Of-An-Exploit.html} \

\bibitem[6]{explarm} Emanuele Acri, \textit{Exploiting Arm Linux Systems}, \url{http://packetstormsecurity.com/files/98376/Exploiting-ARM-Linux-Systems.html} \

\bibitem[7]{fuzzphone} Collin Mulliner, Charlie Miller, \textit{Fuzzing the Phone in your Phone}, Black Hat USA, 2009 \

\bibitem[8]{howtoshell} Jonathan Salwan, \textit{How to Create a Shellcode on ARM Architecture}, \url{http://www.exploit-db.com/papers/15652/} \

\bibitem[9]{telephony} Dustin ,,I)ruid'' Trammel, \textit{Metasploit Framework Telephony}, Black Hat USA, 2009 \

\bibitem[10]{nonexec} Itzhak Avraham, \textit{Non-Executable Stack ARM Exploitation}, Black Hat DC, 2011 \

\bibitem[11]{smashing} jip@soldierx.com, \textit{Stack Smashing On A Modern Linux System}, \url{http://www.soldierx.com/tutorials/Stack-Smashing-Modern-Linux-System} \

\bibitem[12]{metasploit} Metasploit framework, \url{http://www.metasploit.com} \

\bibitem[13]{android} Android project, \url{http://developer.android.com} \

\bibitem[14]{webkit} The WebKit Open Source Project, \url{http://www.webkit.org} \

\end{thebibliography}

\end{document}

